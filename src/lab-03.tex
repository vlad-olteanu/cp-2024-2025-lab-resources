\documentclass{exam}
\usepackage{amssymb}
\usepackage{lipsum} 
\usepackage[
backend=biber
]{biblatex}
\addbibresource{refs.bib}
\usepackage{hyperref}
\hypersetup{
    colorlinks=true,
    linkcolor=blue,
    filecolor=blue,      
    urlcolor=blue,
    citecolor=blue,
}
\newcommand\labnr{3}
\newcommand\lab{Lab \labnr\ - Expressions in C}

\newcommand\uni{Technical University of Cluj-Napoca}
\newcommand\course{Computer Programming}

\newcommand\lvlez{$\bigstar$}
\newcommand\lvlmed{\lvlez\lvlez}
\newcommand\lvlhard{\lvlmed\lvlez}


\pagestyle{headandfoot}
\firstpageheader{}{}{}
\firstpagefootrule
\firstpageheadrule
\firstpagefooter{\sc\uni}{}{\sc\course, Lab \labnr}
\runningheader{\sc\uni}{}{\sc\course, Lab \labnr}
\runningheadrule
\runningfootrule
\runningfooter{}{\thepage}{}

\begin{document}
\begin{center}
   \vspace*{0cm}
   \bfseries\LARGE
   \lab
   \vspace*{1cm}
\end{center}


\begin{questions}
   \question Write a program to calculate the value $z=x^{y}$, for variables x and y of the type double.
   Hint: look up the function pow. \lvlez
   \question Divide 5 by 2 and print the result. First use the integer data type then the float data type. \lvlez

   \question Same as above, but try \textbf{incorrectly} printing the integer result using the \verb|%f| format specifier and the float result using the \verb|%d| format specifier. \lvlez

   \question Write a program to show the number of bytes the C/C++ primitive data types take in the computer memory. Hint: Use \verb|sizeof()|. \lvlez

   \question Write a program to convert a number of seconds into hours, minutes and seconds. If the number of hours goes past 24 it should return to 0. \lvlez

   \question Write a program that reads 2 integers and shows the results of applying all the bitwise operations on them. Print the results using both \verb|%d| and \verb|%x|. \lvlez

   \question Write a program that reads a positive integer number in the range [1600, 4900]. Knowing that that number represents a year, check whether that year is bissextile or not. A year is bissextile if it is divisible by 4, except for years divisible by 100 but not by 400. \lvlmed

   \question Using conditional expressions (\verb|a ? b : c|), write a program which reads a real value for
   x and then computes the
   value for the function: \lvlmed
   \[
      f(x)= \left\{
      \begin{array}{ll}
         x^2-7x+4, & x<-2 \\
         0,        & x=-2 \\
         x^2+5x-2, & x>-2 \\
      \end{array}
      \right.
   \]
   \question Write a program which reads a real number
x, representing a measurement for an angle in Radians,
and then converts it to degrees, minutes, and seconds. \lvlmed

   \question Write a program that reads the value of an angle expressed in degrees and calculates the values for its sine, cosine and tangent. \lvlmed

   \question Check if the cosine of a given angle is equal to 0. \lvlhard

   \question Write a program which reads the integer numbers into variables a, b, c, d and outputs the highest value of fractions a/b and c/d. \lvlmed

   \question Write a program to convert Cartesian coordinates of a given point to polar coordinates. Conversion formulas: \href{https://en.wikipedia.org/wiki/Polar_coordinate_system#Converting_between_polar_and_Cartesian_coordinates}{Converting between polar and Cartesian coordinates}. \lvlhard

\end{questions}

\medskip
\section*{References}
\begin{itemize}
   \item Pb. 1-2, 4, 7-10, 12-13 \cite{cplab03}
\end{itemize}
\printbibliography[heading=none]
\end{document}