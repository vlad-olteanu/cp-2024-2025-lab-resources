\documentclass{exam}
\usepackage{amssymb}
\usepackage{lipsum} 
\usepackage[
backend=biber
]{biblatex}
\addbibresource{refs.bib}
\usepackage{hyperref}
\hypersetup{
    colorlinks=true,
    linkcolor=blue,
    filecolor=blue,      
    urlcolor=blue,
    citecolor=blue,
}
\newcommand\labnr{4}
\newcommand\lab{Lab \labnr\ - Statements in C}

\newcommand\uni{Technical University of Cluj-Napoca}
\newcommand\course{Computer Programming}

\newcommand\lvlez{$\bigstar$}
\newcommand\lvlmed{\lvlez\lvlez}
\newcommand\lvlhard{\lvlmed\lvlez}


\pagestyle{headandfoot}
\firstpageheader{}{}{}
\firstpagefootrule
\firstpageheadrule
\firstpagefooter{\sc\uni}{}{\sc\course, Lab \labnr}
\runningheader{\sc\uni}{}{\sc\course, Lab \labnr}
\runningheadrule
\runningfootrule
\runningfooter{}{\thepage}{}

\begin{document}
\begin{center}
   \vspace*{0cm}
   \bfseries\LARGE
   \lab
   \vspace*{1cm}
\end{center}


\begin{questions}
   \question Real numbers of a sequence of size
   n are read from the standard input. Find and print to
   standard output: the minimum and the maximum values of this sequence, and their positions
   (indices) in the sequence. \lvlez
   \question Given a sequence of
n real numbers sorted in ascending order, verify if a given value,
x, exists in the given sequence, and display this value and its position. \lvlez
   \question Write a program to generate all the prime numbers less than or equal to a natural
   number,
   n. \lvlez
   \question Read from the standard input a natural number,
   n. Find the greatest perfect square
   that is less than or equal to
   n. Then find the least prime number that is greater than or equal to
   n. \lvlez
   \question Read from the standard input a natural number,
   n. Check if this number is palindrome. \lvlmed
   \question Given a matrix of
n $\times$ n elements verify if this matrix is symmetric. \lvlmed

   \question Given 2 lists of integers compute the following:
      \begin{parts}
         \part The intersection of the 2 lists. \lvlez
         \part The union of the 2 lists. \lvlmed
         \part The elements of the first list that are not present in the second list. \lvlez
      \end{parts}   
   \question Read from the standard input the hexadecimal digits of an integer hexadecimal number.
   Find and display the equivalent decimal number. \lvlmed
   \question Read from the standard input the degree and the coefficients of the polynomial $p(x)=a_0+a_1*x^1+a_2*x^2+...+a_n*x^n$.
   Compute and display the value of the polynomial for $x=x_0$ ($x_0$ is read from
   the standard input). \lvlmed
   \question  Write a program to perform the operations +, -, * on two polynomials:
   \[
      A(x) = a_0+a_1*x^1+a_2*x^2+...+a_n*x^n     \]
   \[
      B(x) = b_0+b_1*x^1+b_2*x^2+...+b_n*x^n
   \]
   The degrees and the coefficients are read from the keyboard. \lvlhard
   \question Given a sequence of
n integer numbers, extract the maximum length subsequence
which is in ascending order. \lvlhard
   \question Given a real number a written in base 10, write a program to convert this number in base B, where B $\leq$ 16. \lvlhard

   \question Given a natural number n:
      \begin{parts}
         \part Find the number obtained by eliminating those digits that appear more than once in that number. \lvlhard
         \part Find the number obtained by switching the first digit with the last one, the second with the
next to last one, and so on. \lvlmed
         \part Find the biggest number that could be obtained by a combination of its digits. \lvlhard
      \end{parts}

\end{questions}

\medskip
\section*{References}
\begin{itemize}
   \item Pb. 1-13 \cite{cplab04}
\end{itemize}
\printbibliography[heading=none]
\end{document}