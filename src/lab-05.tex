\documentclass{exam}
\usepackage{amssymb}
\usepackage{lipsum} 
\usepackage[
backend=biber
]{biblatex}
\addbibresource{refs.bib}
\usepackage{hyperref}
\hypersetup{
    colorlinks=true,
    linkcolor=blue,
    filecolor=blue,      
    urlcolor=blue,
    citecolor=blue,
}
\newcommand\labnr{5}
\newcommand\lab{Lab \labnr\ - Functions in C}

\newcommand\uni{Technical University of Cluj-Napoca}
\newcommand\course{Computer Programming}

\newcommand\lvlez{$\bigstar$}
\newcommand\lvlmed{\lvlez\lvlez}
\newcommand\lvlhard{\lvlmed\lvlez}


\pagestyle{headandfoot}
\firstpageheader{}{}{}
\firstpagefootrule
\firstpageheadrule
\firstpagefooter{\sc\uni}{}{\sc\course, Lab \labnr}
\runningheader{\sc\uni}{}{\sc\course, Lab \labnr}
\runningheadrule
\runningfootrule
\runningfooter{}{\thepage}{}

\begin{document}
\begin{center}
   \vspace*{0cm}
   \bfseries\LARGE
   \lab
   \vspace*{1cm}
\end{center}


\begin{questions}   
   \question Write a function which returns the highest perfect square which is less or equal to its
parameter (a positive integer). \lvlez   
   \question Write a function to check if its parameter (positive integer) is a perfect square. Then apply
this function to a vector of positive integers, and extract all perfect squares and place them in
another vector. \lvlmed
   \question Write a function that has the following parameters:
   \begin{itemize}
      \item 2 integers
      \item a pointer to an integer
   \end{itemize}
   The function should add the first 2 numbers and place the result in the integer pointed to by the 3rd parameter. 
   Print the result outside of the function. \lvlez
   \question Write a function that takes as parameters a float \verb|x| and an integer \verb|y| and returns $x^y$. \lvlez
   \question Read an integer representing an amount of money expressed in RON from the standard
input. Write a function to determine the minimum number of banknotes needed to pay that
amount. \lvlez
   \question Write a function which has a string of characters representing a number written using Roman
numerals as a parameter, and returns the corresponding radix (base) 10 Arabian number. \lvlmed
   \question Write the complementary function, which converts a base 10 Arabian number to a number
written with Roman numerals. \lvlmed



   \question Write a function to check whether a character string is a substring of another character
string. The function should return the position at which the substring starts if true, or –1
otherwise. Do not use any functions from \verb|string.h|. \lvlmed
   \question Write the functions for addition, subtraction and multiplication of two matrices, and then
compute
$A=B*C - 2*(B+C)$, where B and C are two $n \times n$ matrices. \lvlhard
   \question Write a function that has the following parameters:
   \begin{itemize}
      \item 2 integers
      \item a function that takes 2 integers and returns an integer
   \end{itemize}
   The function should apply the function given as the third parameter on the first 2 parameters and return the result.
   \lvlhard
   

\end{questions}
\medskip
\section*{References}
\begin{itemize}
   \item Pb. 1-2, 5-9 \cite{cplab05}
\end{itemize}
\printbibliography[heading=none]
\end{document}