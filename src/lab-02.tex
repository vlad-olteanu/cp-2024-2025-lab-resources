\documentclass{exam}
\usepackage{amssymb}
\usepackage{lipsum} 
\usepackage[
backend=biber
]{biblatex}
\addbibresource{refs.bib}

\newcommand\labnr{2}
\newcommand\lab{Lab \labnr\ - Standard I/O in C}

\newcommand\uni{Technical University of Cluj-Napoca}
\newcommand\course{Computer Programming}

\newcommand\lvlez{$\bigstar$}
\newcommand\lvlmed{\lvlez\lvlez}
\newcommand\lvlhard{\lvlmed\lvlez}


\pagestyle{headandfoot}
\firstpageheader{}{}{}
\firstpagefootrule
\firstpageheadrule
\firstpagefooter{\sc\uni}{}{\sc\course, Lab \labnr}
\runningheader{\sc\uni}{}{\sc\course, Lab \labnr}
\runningheadrule
\runningfootrule
\runningfooter{}{\thepage}{}

\begin{document}
\begin{center}
    \vspace*{0cm}
    \bfseries\LARGE
    \lab
    \vspace*{1cm}
\end{center}


\begin{questions}   
   \question Write a program to print an integer read from the standard input as an octal and a hexadecimal
number.  \lvlez 
   \question Write a program to print the number $\pi$ = 3.14159265 using various (floating point) format
descriptors.  \lvlez 
    \question Write a program to check what putch outputs when its argument is a value outside the range
character values [32,126].  \lvlez \
   \question Write a program to print the ASCII codes for the keys of your keyboard. Hint: use printf() with a
proper descriptor for output. \lvlez  
   \question Write a program to print the characters corresponding to the ASCII codes in the range [32,126].   \lvlez 

   \question Write a program to read a lowercase letter string and print the corresponding uppercase letter string.\lvlmed 
   \question Write a program to read a string with only capitals (uppercase letters), and print the corresponding
lowercase letter string. \lvlmed 

\question Write a program to containing invocation(s) of gets(s), where s is an array. Check the contents of
each array member. Hint: use the debugger to set a breakpoint after the invocation of gets() and
inspect the memory area containing the result. Why the newline character ('\textbackslash n') was replaced by '\textbackslash 0'?  \lvlmed
   \question Write a program to calculate the sum, difference, product and quotient of a pair of real numbers.
\textbf{Output the results in a table}, similar to the one below (you do not have to draw lines):   \lvlhard 
\begin{center}
   
\begin{tabular}{|c|c|c|c|c|c|}
   \hline
   x & y & x+y & x-y & x*y & x/y \\
   \hline
   3 & 1.5 & 4.5 & 1.5 & 4.5 & 2 \\
   \hline
\end{tabular}
\end{center}

\question Same as the previous problem, but only show 2 decimals of the numbers and align the numbers to the left. Make each cell 8 characters wide. \lvlhard
\end{questions}

\medskip
\section*{References}
\begin{itemize}
   \item Pb. 1-9 \cite{cplab02}
\end{itemize}
\printbibliography[heading=none]
\end{document}